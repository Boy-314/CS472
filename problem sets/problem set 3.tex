\documentclass{article}
\usepackage{amsmath}
\usepackage{wasysym}
\usepackage[margin=1in]{geometry}
\usepackage{amsthm}
\usepackage{braket}
\usepackage{microtype}
\usepackage{amssymb}
\usepackage{listings}
\usepackage{tikz}
\usetikzlibrary{trees}
\newcommand{\Mod}[1]{\ (\mathrm{mod}\ #1)}
\begin{document}
\title{Artifical Intelligence Problem Set 3}
\date{}
\author{Willie Yee}
\maketitle
\noindent
\textbf{Problem 1.}\\
A. First we calculate $Error(\{A,E\}) = 0 + 6 = 6$. We can perform three different operations on the set:
\begin{enumerate}
	\item Add an object to the set. If we add B, then $Error(\{A,E,B\}) = 3 + 0 = 3$. If we add C, then $Error(\{A,E,C\}) = 2 + 0 = 2$. If we add D, then $Error(\{A,E,D\}) = 2 + 0 = 2$.
	\item Delete an object from the set. If we delete E, then $Error(\{A\}) = 0 + 10 = 10$. If we delete A, then $Error(\{E\}) = 0 + 16 = 16$. In either case, this is higher error than what we had before.
	\item Swap out an object in the set with an object outside the set. If we swap out A and add B, then $Error(\{B,E\}) = 0 + 8 = 8$. If we swap out A and add C, then $Error(\{C,E\}) = 0 + 9 = 9$. If we swap out A and add D, then $Error(\{D,E\}) = 0 + 10 = 10$. If we swap out E and add B, then $Error(\{A,B\}) = 2 + 2 = 4$. If we swap out E and add C, then $Error(\{A,C\}) = 1 + 3 = 4$. If we swap out E and add D, then $Error(\{A,D\}) = 1 + 4 = 5$.
\end{enumerate}
Out of all of these possibilities, the operation with the lowest error after the operation would be to add C or add D to the set. To determine what happens at the next iteration, we must assume that we add C to the set, and not D. Thus, our set as of now is $\{A,C,E\}$. With this, the next iteration will be the set $\{B,C,E\}$ where we have swapped out A and added B to our set. $Error(\{B,C,E\}) = 0 + 1 = 1$.\\
\\
B. The size of the state space is all combinations of 1, 2, 3, ... , N objects. Thus, the state space is $${N\choose 1}+{N\choose 2}+\cdots +{N\choose N}=\sum_{i=1}^{N} {N\choose i}.$$
The maximal number of neighbors of any state can be calculated by the total number of operations that we can perform on a general state. Suppose we have a state with $X$ elements in it. We can add $N-X$ elements to the state space. We can remove any one of $X$ elements from the state space. Or we can replace any of the $X$ elements with one of the $N-X$ elements not in the space. This totals to 
\begin{align*}
	N-X + X + X\cdot (N-X) &= N + X\cdot (N-X)\\
	&= N+XN-X^2
\end{align*}
To maximize that, we look where the derivative of $N+XN-X^2=0$ with respect to $X$: $-2X+N=0$ implies that $X=\frac{N}{2}$. Thus the maximum number of neighbors for any given state is 
\begin{align*}
	N + \frac{N}{2} \cdot N - (\frac{N}{2})^2 &= N+\frac{N^2}{2} - \frac{N^2}{4}\\
	&=\frac{N^2+4N}{4}
\end{align*}
\textbf{Problem 2.} Roots that are pruned (in order):
\begin{enumerate}
	\item 6, the leaf node
	\item 20, the leaf node
	\item 7, the leaf node
\end{enumerate}
The best move for the max node at the top level is to go right for a score of 8.
\end{document}